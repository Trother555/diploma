\subsection{Криптография}
Шифр – это пятёрка $\{X, Y, K, D, E\}$ (\autocite{Drs} с. 1), где:
\begin{enumerate}
	\item $X, Y, K$ --- соответственно множества открытых текстов, шифротекстов и ключей (часто полагают $X = Y = K$).\\
	\item $E_{k}$ --- функция зашифрования на ключе $k$, $D_{k}$ --- функция расшифрования ключом $k$. $E : X \times K \rightarrow Y$, $D : Y \times K \rightarrow X$. Причём для любых $x \in X и k \in K$справедливо:
	$$D_{k}(E_{k}(x)) = x$$
\end{enumerate}

Назначение шифра --- преобразовывать открытые тексты таким образом, чтобы из преобразованных текстов них нельзя было извлечь никакой информации об исходных текстах. 
Шифр называется коммутативным, если для любых $k_{1},k_{2}\in K и x  \in X$выполняется:
		$$E_{k_{1}}(E_{k_{2}}(x)) = E_{k_{2}}(E_{k_{1}}(x))$$
		
Атака <<человек посередине>> (\autocite{Shnayer} глава 3, параграф 1), --- тип атаки на криптографические системы, при которых активный злоумышленник подменяет в канале своими сообщениями сообщения общающихся сторон.

ЭЦП (Электронная цифровая подпись)(\autocite{Shnayer} глава 4), --- система алгоритмов, при помощи которых стороны могут подписывать свои сообщения, удостоверяя свою личность. ЭЦП могут быть заверены третьим лицом, называемым центром сертификации, для надёжности их использования.

Если стороны хотят установить защищённое соединение, то они могут вместе выработать ключ, называемый сеансовым, а затем использовать этот ключ для шифрования и расшифрования сообщений. Для решения задачи выработки ключа существует несколько протоколов, среди которых в настоящее время наиболее распространены протоколы RSA (англ. Rivest, Shamir и Adleman) и Диффи-Хеллмана (англ. Diffie-Hellman). 
Помимо указанных протоколов существуют также и другие способы решения задачи согласования ключа. Один из них – трёхпроходной протокол Шамира. Ниже, среди других протоколов, описан принцип его работы.
\subsection{Кошка Арнольда}
Пусть область $D\subset \mathbb{R}$ яляется квадратом со стороной $1$, левый нижний край которого расположен в начале координат, а правая и верхняя стороны которого отождествлены соответственно с левой и нижней стороной. Область $D$ можно  также рассматривать как тор.

Рассмотрим на $D$ отображение:
$$\Gamma_{cat}(x,y) = \begin{bmatrix} 1 & 1 \\ 1 & 2 \end{bmatrix}\begin{bmatrix} x \\ y \end{bmatrix} (mod \quad 1)$$
Это отображение --- гиперболический торальный автоморфизм, называемый <<Кошкой Арнольда>>. 

Для целей этой работы мы будем применять <<Кошку>> к квадратным изоображениям, которые в памяти компьютера представлены как набор точек заданного цвета с целочисленными координатами. Понадобится дискретный аналог описанного отображения, называемый <<Дискретной кошкой Арнольда>>.
Пусть координаты дискретного ихображения $x,y$ принимают целые значения от $0$ до $N$. Тогда отображение <<Дискретная кошка Арнольда>> определим следующим образом:
$$\Gamma_{A}(x,y) = \begin{bmatrix} 1 & 1 \\ 1 & 2 \end{bmatrix}\begin{bmatrix} x \\ y \end{bmatrix} (mod \quad N)$$
Описанные выше отображения обладают рядом свойств. Приведём здесь некоторые из них.
\begin{myproperty}Пусть $D$ --- квадрат размера $N$ на $N$ с целочисленными координатами, тогда существует такое $m$, что:
$$\Gamma_{A}^{m}(D) = D$$
\end{myproperty}
Это свойство описано в \autocite{cat}. Если, например, $N = 50$, то период $m = 150$.
Благодаря следующему свойству, описанному в \autocite{arn}, отображение <<Кошка Арнольда>> применимо в криптографических целях. 
\begin{myproperty}Пусть $B$ --- какая-нибудь часть квадрата $D$. Тогда доля площади фигуры $\Gamma_{cat}^{k}(D)$, находящаяся в $B$, будет пропорциональна площади $B$ при $k\rightarrow\inf$;$$$$
\end{myproperty}
Это означает, что изоражение будет равномерно  <<размазываться>> по квадрату.