Шифр – это пятёрка $\{X, Y, K, D, E\}$ \autocite{Drs}( с. 1), где:
\begin{enumerate}
	\item $X, Y, K$ --- соответственно множества открытых текстов, шифротекстов и ключей (часто полагают $X = Y = K$).\\
	\item $E_{k}$ --- функция зашифрования на ключе $k$, $D_{k}$ --- функция расшифрования ключом $k$. $E : X \times K \rightarrow Y$, $D : Y \times K \rightarrow X$. Причём для любых $x \in X и k \in K$справедливо:
	$$D_{k}(E_{k}(x)) = x$$
\end{enumerate}

Назначение шифра --- преобразовывать открытые тексты таким образом, чтобы из преобразованных текстов них нельзя было извлечь никакой информации об исходных текстах. 
Шифр называется коммутативным, если для любых $k_{1},k_{2}\int K и x  \in X$выполняется:
		$$E_{k_{1}}(E_{k_{2}}(x)) = E_{k_{2}}(E_{k_{1}}(x))$$
		
Атака человек посередине \autocite{Shnayer}(глава 3, параграф 1), – тип атаки на криптографические системы, при которых активный злоумышленник подменяет в канале своими сообщениями сообщения общающихся сторон.

ЭЦП (Электронная цифровая подпись)\autocite{Shnayer}( глава 4), --- система алгоритмов, при помощи которых стороны могут подписывать свои сообщения, удостоверяя свою личность. ЭЦП могут быть заверены третьим лицом, называемым центром сертификации, для надёжности их использования.

Если стороны хотят установить защищённое соединение, то они могут вместе выработать ключ, называемый сеансовым, а затем использовать этот ключ для шифрования и расшифрования сообщений. Для решения задачи выработки ключа существует несколько протоколов, среди которых в настоящее время наиболее распространены протоколы RSA (англ. Rivest, Shamir и Adleman) и Диффи-Хеллмана (англ. Diffie-Hellman). 
Помимо указанных протоколов существуют также и другие способы решения задачи согласования ключа. Один из них – трёхпроходной протокол Шамира. Ниже, среди других протоколов, описан принцип его работы.
