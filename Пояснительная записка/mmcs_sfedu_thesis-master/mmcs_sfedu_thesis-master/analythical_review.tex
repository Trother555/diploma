
\subsection{Трёхпроходной протокол Шамира}
Трёхпроходной протокол Шамира \autocite{Shnayer}(глава 3, параграф 1) позволяет одной стороне послать другой стороне зашифрованное сообщение без необходимости заранее распространять ключи.\\
\textbf{Протокол 1 (Шамира)}\\
\hspace*{10mm}\textbf{Вход:} коммутативный шифр $\{X, Y, K, D, E\}$ для шифрования сеансового ключа, шифр $\{X', Y', K', D', E'\}$ для шифрования сообщений.\\
Первая сторона: сеансовый ключ $s\in K',$ ключ $k_{1}\in K.$\\
Вторая сторона: ключ $k_{2}\in K$.\\
\hspace*{10mm}\textbf{Выход:}вторая сторона: сеансовый ключ $s\in K'$.
\begin{enumerate}
	\item Первая сторона посылает второй сообщение $m_{1} = E_{k_{1}}(s).$
	\item Вторая сторона принимает $m_{1}$ и посылает первой $m_{2} = E_{k_{2}}(m1)$.
	\item Первая сторона применяет алгоритм расшифрования к $m_{2}$ и посылает второй $m_{3} = D_{k_{1}}(m_{2})$.
	\item Вторая сторона расшифровывает $s = D_{k_{2}}(m_{3})$.
\end{enumerate}
Итак, обе стороны получают $s$ - секретный ключ.
\subsubsection{Безопасность}
Стойкость протокола при противодействии пассивному злоумыщленнику опирается на стойкость шифра, исопльзуемого для зашифрования сеансового ключа. Протокол уязвим против активного злоумышленника при атаке <<человек посередине>>.
\subsubsection{Шифр для трёхпроходного протокола Шамира}
Этот шифр описан в \cite{Shnayer}(22.3). Пусть $p$ - большое простое число (его должны знать оба участника обмена информацией, но оно не является секретом). Выберем $e$ - взаимно простое с $p-1$. Вычислим $d \equiv e^{-1}(mod\ p-1)$. Пусть $m$ - исходное сообщение - целое число: $0<m<p$, тогда алгоритм зашифрования:
$$c=m^{e}mod\ p$$
Алгоритм расшифрования:
$$m=c^{d}mod\ p$$
Безопасность этого шифра основана на проблеме дискретного логарифмирования.
\subsection{Протокол Диффи-Хеллмана}
Протокол Диффи-Хеллмана \autocite{Shnayer}(Глава 22 параграф 1) позволяет сторонам выработать общий секрет ключ.\\
\textbf{Протокол 2 (Диффи-Хеллмана)}\\
\hspace*{10mm}\textbf{Вход:} Большие простые числа $n$ и $g$ такие, что:
$$g^{\phi(n)}\equiv1(mod\ n),$$
Где $\phi(n)$ --- функция Эйлера, равная количеству простых чисел, менишьх $n$.
Первая сторона: случайное большое целое число $x$\\
Вторая сторона: случайное большое целое число $y$\\
\hspace*{10mm}\textbf{Выход:} вторая сторона: сеансовый ключ $k$.\\
\hspace*{10mm}Первая сторона: сеансовый ключ $k$.
\begin{enumerate}
	\item Первая посылает второй:
	$$X = g^{x}mod\ n.$$
	\item Вторая посылает первой:
	$$Y = g^{y}mod\ n.$$
	\item Первая сторона вычисляет:
	$$k = Y^{x}mod\ n.$$
	\item Вторая сторона вычисляет:
	$$k = X^{y}mod\ n$$
\end{enumerate}
Итак, обе стороны получают $k$ - сеансовый ключ.
\subsubsection{Безопасность}
Стойкость протокола при противодействии пассивному злоумыщленнику опирается на проблему дискретного логарифмирования. Протокол уязвим против активного злоумышленника при атаке <<человек посередине>>.
\subsection{Протокол RSA}
Протокол RSA \autocite{Shnayer}(глава 19, параграф 2) позволяет одной стороне послать зашифрованное сообщение при условии, что ей известен открытый ключ принимающей стороны. Сначала принимающая сторона должна сгенерировать пару --- закрытый и открытый ключ.\\
\textbf{Алгоритм 1 (Генерация пары закрытый-открытый ключ)}\\
\hspace*{10mm}\textbf{Вход:} числа $p$ и $q$. Для лучшей безопасности - равной длины \autocite{Shnayer}(19.3).\\
\hspace*{10mm}\textbf{Выход:} пара ключей $(e,d).$ 
\begin{enumerate}
	\item Вычислить:
	$$n = pq$$
	\item Взять случайным образом $e$ так, чтобы выполнялось:
	$$(e,\phi(n)) = 1.$$
	Это открытый ключ.
	\item Вычислить, используя свойство $\phi(n)=(p-1)(q-1)$:
	$$d=e^{-1}mod(\phi(n)) = e^{-1}mod((p-1)(q-1)).$$
	Это закрытый ключ.
\end{enumerate}
Теперь можно использовать следующий протокол выработки сеансового ключа:\\
\textbf{Протокол 3 (RSA)}\\
\hspace*{10mm}\textbf{Вход:}
Первая сторона: закрытый ключ RSA: $d$\\
Вторая сторона: открытый ключ RSA: $e$, сеансовый ключ $s$\\
\hspace*{10mm}\textbf{Выход:} первая сторона: сеансовый ключ $s$.\\
\begin{enumerate}
	\item Вторая посылает первой:
	$$m_{1} = s^{e}mod\ n.$$
	\item Первая сторона расшифровывает:
	$$s = m_{1}^{d}mod\ n.$$
\end{enumerate}
\subsubsection{Безопасность}
Стойкость протокола при противодействии пассивному злоумыщленнику опирается на проблему разложения числа на простые множители. Протокол уязвим против активного злоумышленника при атаке <<человек посередине>>.