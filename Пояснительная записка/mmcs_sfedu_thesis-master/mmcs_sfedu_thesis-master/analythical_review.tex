
\subsection{Трёхпроходной протокол Шамира}
Трёхпроходной протокол Шамира (\autocite{Shnayer} глава 3, параграф 1) позволяет одной стороне послать другой стороне сеансовый ключ без необходимости заранее распространять ключи.\\
\textbf{Протокол 1 (Шамира)}\\
\hspace*{10mm}\textbf{Вход:} коммутативный шифр $\{X, Y, K, D, E\}$ для шифрования сеансового ключа, шифр $\{X', Y', K', D', E'\}$ для шифрования сообщений.\\
Первая сторона: сеансовый ключ $s\in K',$ ключ $k_{1}\in K.$\\
Вторая сторона: ключ $k_{2}\in K$.\\
\hspace*{10mm}\textbf{Выход:}вторая сторона: сеансовый ключ $s\in K'$.
\begin{enumerate}
	\item Первая сторона посылает второй сообщение $m_{1} = E_{k_{1}}(s).$
	\item Вторая сторона принимает $m_{1}$ и посылает первой $m_{2} = E_{k_{2}}(m_{1})$.
	\item Первая сторона применяет алгоритм расшифрования к $m_{2}$ и посылает второй $m_{3} = D_{k_{1}}(m_{2})$.
	\item Вторая сторона расшифровывает $s = D_{k_{2}}(m_{3})$.
\end{enumerate}
Итак, обе стороны получают $s$ - секретный ключ.
\subsubsection{Безопасность}
Стойкость протокола при противодействии пассивному злоумыщленнику опирается на стойкость шифра, исопльзуемого для зашифрования сеансового ключа. Протокол уязвим против активного злоумышленника при атаке <<человек посередине>> (\autocite{Shnayer}, глава 22, пункт 3). Действительно, злоумышленник может, вмешавшись в протокол, выполнить его по отдельности с обеими сторонами:\\
\textbf{Протокол 4 (Шамира с вмешательством активного злоумышленника)}\\
\hspace*{10mm}\textbf{Вход:} коммутативный шифр $\{X, Y, K, D, E\}$ для шифрования сеансового ключа, шифр $\{X', Y', K', D', E'\}$ для шифрования сообщений.\\
Первая сторона: сеансовый ключ $s\in K',$ ключ $k_{1}\in K.$\\
Вторая сторона: ключ $k_{2}\in K$.\\
Злоумышленник: сеансовый ключ $s'\in K',$  ключи $k_{1}',k_{2}'\in K.$\\
\hspace*{10mm}\textbf{Выход:} вторая сторона: сеансовый ключ $s'\in K'$.\\
Злоумышленник: сеансовый ключ $s\in K'.$
\begin{enumerate}
	\item Первая сторона посылает второй сообщение $m_{1} = E_{k_{1}}(s)$
	\begin{enumerate}
		\item Злоумышленник посылает второй стороне $m_{1}' = E_{k_{1}}(s).$
		\item Злоумышленник перехватывает $m_{1}$ и посылает первой стороне:
		$$m_{2}' = E_{k_{2}'}(m_{1})$$
	\end{enumerate}
	\item Вторая сторона принимает $m_1'$ и посылает первой $m_{2}''= E_{k_{2}} (m_{1}').$
	\begin{enumerate}
		\item Злоумышленник перехватывает $m_{2}''$ и посылает второй стороне:
		$$m_{3}'  = D_{k_{1}}(m_{2}'').$$
	\end{enumerate}
	\item Первая сторона применяет алгоритм расшифрования к $m_{2}'$ и посылает второй
	$$m_{3}'' = D_{k_{1}}(m_{2}')$$
	\begin{enumerate}
		\item Злоумышленник перехватывает $m_{3}''$ и получает $s = D_{k_{2}} (m_{3}'')$.
	\end{enumerate}
	\item Вторая сторона расшифровывает $s' = D_{k_{2}}(m_{3}')$.
\end{enumerate}
Таким образом, стороны думают, что договорились о ключе s друг с другом, на самом же деле они договорились о ключах $s$ и $s'$ со злоумышленником, который теперь может выступать посредником в их общении и читать всю переписку.
\subsubsection{Шифр для трёхпроходного протокола Шамира}
Этот шифр описан в (\cite{Shnayer} глава 22, пункт 3). Пусть $p$ - большое простое число (его должны знать оба участника обмена информацией, но оно не является секретом). Выберем $e$ - взаимно простое с $p-1$. Вычислим $d \equiv e^{-1}(mod\ p-1)$. Пусть $m$ - исходное сообщение - целое число: $0<m<p$, тогда алгоритм зашифрования:
$$c=m^{e}mod\ p$$
Алгоритм расшифрования:
$$m=c^{d}mod\ p$$
Безопасность этого шифра основана на проблеме дискретного логарифмирования.

Покажем, что описанный шифр является коммутативным, как того требует протокол Шамира. Пусть $m$ --- шифруемое сообщение, а пары чисел $e_{1}, d_{1}$ и $e_{2}, d_{2}$ удовлетворяют условиям:
\begin{gather*}
	d_{1} \equiv e_{1}^{-1}mod\ (p-1)\\
	d_{2} \equiv e_{2}^{-1}mod\ (p-1)
\end{gather*}
Нужно показать, что справедливо:
$$
E_{e_{2}}((E_{e_{1}}(m)) = E_{e_{1}}((E_{e_{2}}(m))
$$
В самом деле:
\begin{gather*}
E_{e_{2}}((E_{e_{1}}(m)) = E_{e_{2}}(m^{e_{1}}mod\ (p-1))=\\
=m^{e_{1}\cdot{}e_{2}}mod\ (p-1) = m^{e_{2}\cdot{}e_{1}}mod\ (p-1)=\\
 = E_{e_{1}}((E_{e_{2}}(m))
\end{gather*}
\subsection{Протокол Диффи-Хеллмана}
Протокол Диффи-Хеллмана (\autocite{Shnayer} глава 22, параграф 1) позволяет сторонам выработать общий секрет ключ.\\
\textbf{Протокол 2 (Диффи-Хеллмана)}\\
\hspace*{10mm}\textbf{Вход:} Большие простые числа $n$ и $g$ такие, что:
$$g^{\phi(n)}\equiv1(mod\ n),$$
где $\phi(n)$ --- функция Эйлера, равная количеству простых чисел, менишьх $n$.\\
Первая сторона: случайное большое целое число $x$\\
Вторая сторона: случайное большое целое число $y$\\
\hspace*{10mm}\textbf{Выход:} вторая сторона: сеансовый ключ $k$.\\
\hspace*{10mm}Первая сторона: сеансовый ключ $k$.
\begin{enumerate}
	\item Первая посылает второй:
	$$X = g^{x}mod\ n.$$
	\item Вторая посылает первой:
	$$Y = g^{y}mod\ n.$$
	\item Первая сторона вычисляет:
	$$k = Y^{x}mod\ n.$$
	\item Вторая сторона вычисляет:
	$$k = X^{y}mod\ n$$
\end{enumerate}
Итак, обе стороны получают $k$ - сеансовый ключ.
\subsubsection{Безопасность}
Стойкость протокола при противодействии пассивному злоумыщленнику опирается на проблему дискретного логарифмирования. Протокол уязвим против активного злоумышленника при атаке <<человек посередине>>. Схема атаки та же, что и в протоколе Шамира.
\subsection{Протокол RSA}
Протокол RSA (\autocite{Shnayer} глава 19, параграф 2) позволяет одной стороне послать зашифрованное сообщение при условии, что ей известен открытый ключ принимающей стороны. Сначала принимающая сторона должна сгенерировать пару --- закрытый и открытый ключ.\\
\textbf{Алгоритм 1 (Генерация пары закрытый-открытый ключ)}\\
\hspace*{10mm}\textbf{Вход:} числа $p$ и $q$. Для лучшей безопасности - равной длины.\\
\hspace*{10mm}\textbf{Выход:} пара ключей $(e,d).$ 
\begin{enumerate}
	\item Вычислить:
	$$n = pq$$
	\item Взять случайным образом $e$ так, чтобы выполнялось:
	$$(e,\phi(n)) = 1.$$
	Это открытый ключ.
	\item Вычислить, используя свойство $\phi(n)=(p-1)(q-1)$:
	$$d=e^{-1}\ mod(\phi(n)) = e^{-1}\ mod((p-1)(q-1)).$$
	Это закрытый ключ.
\end{enumerate}
Теперь можно использовать следующий протокол выработки сеансового ключа:\\
\textbf{Протокол 3 (RSA)}\\
\hspace*{10mm}\textbf{Вход:}
Первая сторона: закрытый ключ RSA: $d$\\
Вторая сторона: открытый ключ RSA: $e$, сеансовый ключ $s$\\
\hspace*{10mm}\textbf{Выход:} первая сторона: сеансовый ключ $s$.\\
\begin{enumerate}
	\item Вторая сторона посылает первой:
	$$m_{1} = s^{e}\ mod\ n.$$
	\item Первая сторона расшифровывает:
	$$s = m_{1}^{d}\ mod\ n.$$
\end{enumerate}
\subsubsection{Безопасность}
Стойкость протокола при противодействии пассивному злоумыщленнику опирается на проблему разложения числа на простые множители. Протокол уязвим против активного злоумышленника при атаке <<человек посередине>>. Схема атаки та же, что и в протоколе Шамира.