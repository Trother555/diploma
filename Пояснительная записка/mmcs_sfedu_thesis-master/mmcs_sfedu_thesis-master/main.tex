% В этом файле следует писать текст работы, разбивая его на
% разделы (section), подразделы (subsection) и, если нужно,
% главы (chapter).

% Предварительно следует указать необходимую информацию
% в файле SETUP.tex

\input{preamble.tex}

\usepackage{csquotes}        % Оформление списка литературы
\usepackage[
%backend=biber,
%hyperref=auto,
%sorting=none, % сортировка в порядке встречаемости ссылок
%language=auto,
%citestyle=gost-numeric,
%bibstyle=gost-numeric
]{biblatex}
\addbibresource{biblio.bib} % Файл с лит.источниками


\NewBibliographyString{langjapanese}
\NewBibliographyString{fromjapanese}

\begin{document}

\Intro

В работе приводится описание трёхпроходного протокола Шамира, сравнение его с протоколами Диффи-Хеллмана, RSA и их различными модификациями. Предлагается способ защиты протокола от атаки типа "человек в середине"\ и его практическая реализация.
\subsection*{Актуальность}
В настоящее время существует определённое количество криптографических протоколов, которые позволяют сторонам получить разделяемый секрет - сеансовый ключ. Однако далеко не все они имеют практическую реализацию, которая, ввиду различий этих протоколов в способах шифрования, количестве и смысле передаваемой информации и т.д., может представлять определённый интерес. Ведь в зависимости от условий, при которых осуществляется защищённый обмен информацией, выбор того или иного протокола может оказаться более удачным - это делает задачу создания таких реализаций актуальной.
\subsection*{Новизна}
На данный момент в таких известных криптографических библиотеках, как OpenSSL и NSS, отсутствует трёхпроходной протокол Шамира - схема защищённой пересылки сообщения (в т.ч. и сеансового ключа), которую я реализовал в ходе своей работы.

% Если typeOfWork в SETUP.tex задан как 2 или 3, то начинать
% надо не с section (раздел), а с главы (chapter)

\section{Необходимые сведения}
\subsection{Криптография}
Шифр – это пятёрка $\{X, Y, K, D, E\}$ (\autocite{Drs} с. 1), где:
\begin{enumerate}
	\item $X, Y, K$ --- соответственно множества открытых текстов, шифротекстов и ключей (часто полагают $X = Y = K$).\\
	\item $E_{k}$ --- функция зашифрования на ключе $k$, $D_{k}$ --- функция расшифрования ключом $k$. $E : X \times K \rightarrow Y$, $D : Y \times K \rightarrow X$. Причём для любых $x \in X и k \in K$справедливо:
	$$D_{k}(E_{k}(x)) = x$$
\end{enumerate}

Назначение шифра --- преобразовывать открытые тексты таким образом, чтобы из преобразованных текстов них нельзя было извлечь никакой информации об исходных текстах. 
Шифр называется коммутативным, если для любых $k_{1},k_{2}\in K и x  \in X$выполняется:
		$$E_{k_{1}}(E_{k_{2}}(x)) = E_{k_{2}}(E_{k_{1}}(x))$$
		
Атака <<человек посередине>> (\autocite{Shnayer} глава 3, параграф 1), --- тип атаки на криптографические системы, при которых активный злоумышленник подменяет в канале своими сообщениями сообщения общающихся сторон.

ЭЦП (Электронная цифровая подпись)(\autocite{Shnayer} глава 4), --- система алгоритмов, при помощи которых стороны могут подписывать свои сообщения, удостоверяя свою личность. ЭЦП могут быть заверены третьим лицом, называемым центром сертификации, для надёжности их использования.

Если стороны хотят установить защищённое соединение, то они могут вместе выработать ключ, называемый сеансовым, а затем использовать этот ключ для шифрования и расшифрования сообщений. Для решения задачи выработки ключа существует несколько протоколов, среди которых в настоящее время наиболее распространены протоколы RSA (англ. Rivest, Shamir и Adleman) и Диффи-Хеллмана (англ. Diffie-Hellman). 
Помимо указанных протоколов существуют также и другие способы решения задачи согласования ключа. Один из них – трёхпроходной протокол Шамира. Ниже, среди других протоколов, описан принцип его работы.
\subsection{Кошка Арнольда}
Пусть область $D\subset \mathbb{R}$ яляется квадратом со стороной $1$, левый нижний край которого расположен в начале координат, а правая и верхняя стороны которого отождествлены соответственно с левой и нижней стороной. Область $D$ можно  также рассматривать как тор.

Рассмотрим на $D$ отображение:
$$\Gamma_{cat}(x,y) = \begin{bmatrix} 1 & 1 \\ 1 & 2 \end{bmatrix}\begin{bmatrix} x \\ y \end{bmatrix} (mod \quad 1)$$
Это отображение --- гиперболический торальный автоморфизм, называемый <<Кошкой Арнольда>>. 

Для целей этой работы мы будем применять <<Кошку>> к квадратным изоображениям, которые в памяти компьютера представлены как набор точек заданного цвета с целочисленными координатами. Понадобится дискретный аналог описанного отображения, называемый <<Дискретной кошкой Арнольда>>.
Пусть координаты дискретного ихображения $x,y$ принимают целые значения от $0$ до $N$. Тогда отображение <<Дискретная кошка Арнольда>> определим следующим образом:
$$\Gamma_{A}(x,y) = \begin{bmatrix} 1 & 1 \\ 1 & 2 \end{bmatrix}\begin{bmatrix} x \\ y \end{bmatrix} (mod \quad N)$$
Описанные выше отображения обладают рядом свойств. Приведём здесь некоторые из них.
\begin{myproperty}Пусть $D$ --- квадрат размера $N$ на $N$ с целочисленными координатами, тогда существует такое $m$, что:
$$\Gamma_{A}^{m}(D) = D$$
\end{myproperty}
Это свойство описано в \autocite{cat}. Если, например, $N = 50$, то период $m = 150$.
Благодаря следующему свойству, описанному в \autocite{arn}, отображение <<Кошка Арнольда>> применимо в криптографических целях. 
\begin{myproperty}Пусть $B$ --- какая-нибудь часть квадрата $D$. Тогда доля площади фигуры $\Gamma_{cat}^{k}(D)$, находящаяся в $B$, будет пропорциональна площади $B$ при $k\rightarrow\inf$;$$$$
\end{myproperty}
Это означает, что изоражение будет равномерно  <<размазываться>> по квадрату.

\section{Аналитический обзор}
В этом разделе представлено краткое описание протоколов, которые будут в дальнейшем рассматриваться.

\subsection{Трёхпроходной протокол Шамира}
Трёхпроходной протокол Шамира \autocite{Shnayer}(глава 3, параграф 1) позволяет одной стороне послать другой стороне зашифрованное сообщение без необходимости заранее распространять ключи.\\
\textbf{Протокол 1 (Шамира)}\\
\hspace*{10mm}\textbf{Вход:} коммутативный шифр $\{X, Y, K, D, E\}$ для шифрования сеансового ключа, шифр $\{X', Y', K', D', E'\}$ для шифрования сообщений.\\
Первая сторона: сеансовый ключ $s\in K',$ ключ $k_{1}\in K.$\\
Вторая сторона: ключ $k_{2}\in K$.\\
\hspace*{10mm}\textbf{Выход:}вторая сторона: сеансовый ключ $s\in K'$.
\begin{enumerate}
	\item Первая сторона посылает второй сообщение $m_{1} = E_{k_{1}}(s).$
	\item Вторая сторона принимает $m_{1}$ и посылает первой $m_{2} = E_{k_{2}}(m1)$.
	\item Первая сторона применяет алгоритм расшифрования к $m_{2}$ и посылает второй $m_{3} = D_{k_{1}}(m_{2})$.
	\item Вторая сторона расшифровывает $s = D_{k_{2}}(m_{3})$.
\end{enumerate}
Итак, обе стороны получают $s$ - секретный ключ.
\subsubsection{Безопасность}
Стойкость протокола при противодействии пассивному злоумыщленнику опирается на стойкость шифра, исопльзуемого для зашифрования сеансового ключа. Протокол уязвим против активного злоумышленника при атаке <<человек посередине>>.
\subsubsection{Шифр для трёхпроходного протокола Шамира}
Этот шифр описан в \cite{Shnayer}(22.3). Пусть $p$ - большое простое число (его должны знать оба участника обмена информацией, но оно не является секретом). Выберем $e$ - взаимно простое с $p-1$. Вычислим $d \equiv e^{-1}(mod\ p-1)$. Пусть $m$ - исходное сообщение - целое число: $0<m<p$, тогда алгоритм зашифрования:
$$c=m^{e}mod\ p$$
Алгоритм расшифрования:
$$m=c^{d}mod\ p$$
Безопасность этого шифра основана на проблеме дискретного логарифмирования.
\subsection{Протокол Диффи-Хеллмана}
Протокол Диффи-Хеллмана \autocite{Shnayer}(Глава 22 параграф 1) позволяет сторонам выработать общий секрет ключ.\\
\textbf{Протокол 2 (Диффи-Хеллмана)}\\
\hspace*{10mm}\textbf{Вход:} Большие простые числа $n$ и $g$ такие, что:
$$g^{\phi(n)}\equiv1(mod\ n),$$
Где $\phi(n)$ --- функция Эйлера, равная количеству простых чисел, менишьх $n$.
Первая сторона: случайное большое целое число $x$\\
Вторая сторона: случайное большое целое число $y$\\
\hspace*{10mm}\textbf{Выход:} вторая сторона: сеансовый ключ $k$.\\
\hspace*{10mm}Первая сторона: сеансовый ключ $k$.
\begin{enumerate}
	\item Первая посылает второй:
	$$X = g^{x}mod\ n.$$
	\item Вторая посылает первой:
	$$Y = g^{y}mod\ n.$$
	\item Первая сторона вычисляет:
	$$k = Y^{x}mod\ n.$$
	\item Вторая сторона вычисляет:
	$$k = X^{y}mod\ n$$
\end{enumerate}
Итак, обе стороны получают $k$ - сеансовый ключ.
\subsubsection{Безопасность}
Стойкость протокола при противодействии пассивному злоумыщленнику опирается на проблему дискретного логарифмирования. Протокол уязвим против активного злоумышленника при атаке <<человек посередине>>.
\subsection{Протокол RSA}
Протокол RSA \autocite{Shnayer}(глава 19, параграф 2) позволяет одной стороне послать зашифрованное сообщение при условии, что ей известен открытый ключ принимающей стороны. Сначала принимающая сторона должна сгенерировать пару --- закрытый и открытый ключ.\\
\textbf{Алгоритм 1 (Генерация пары закрытый-открытый ключ)}\\
\hspace*{10mm}\textbf{Вход:} числа $p$ и $q$. Для лучшей безопасности - равной длины \autocite{Shnayer}(19.3).\\
\hspace*{10mm}\textbf{Выход:} пара ключей $(e,d).$ 
\begin{enumerate}
	\item Вычислить:
	$$n = pq$$
	\item Взять случайным образом $e$ так, чтобы выполнялось:
	$$(e,\phi(n)) = 1.$$
	Это открытый ключ.
	\item Вычислить, используя свойство $\phi(n)=(p-1)(q-1)$:
	$$d=e^{-1}mod(\phi(n)) = e^{-1}mod((p-1)(q-1)).$$
	Это закрытый ключ.
\end{enumerate}
Теперь можно использовать следующий протокол выработки сеансового ключа:\\
\textbf{Протокол 3 (RSA)}\\
\hspace*{10mm}\textbf{Вход:}
Первая сторона: закрытый ключ RSA: $d$\\
Вторая сторона: открытый ключ RSA: $e$, сеансовый ключ $s$\\
\hspace*{10mm}\textbf{Выход:} первая сторона: сеансовый ключ $s$.\\
\begin{enumerate}
	\item Вторая посылает первой:
	$$m_{1} = s^{e}mod\ n.$$
	\item Первая сторона расшифровывает:
	$$s = m_{1}^{d}mod\ n.$$
\end{enumerate}
\subsubsection{Безопасность}
Стойкость протокола при противодействии пассивному злоумыщленнику опирается на проблему разложения числа на простые множители. Протокол уязвим против активного злоумышленника при атаке <<человек посередине>>.
\section{Теоретическая часть}
Сравним теперь протоколы Шамира и Диффи-Хеллмана и рассмотрим вопрос о модификации протокола Шамира для защиты от атаки типа <<человек посередине>>.

Также как протоколы RSA и Диффи-Хеллман, протокол Шамира уязвим к атаке человек посередине. 
\subsection{Способ защиты описанного протокола}
Источник уязвимости упомянутых протоколов кроется в том, что они не предоставляют авторизации, то есть у сторон нет никакого способа проверить, что они действительно общаются друг с другом, во время выполнения протокола. Для разрешения проблемы авторизации может быть использована заверенная ЭЦП.

Возможны различные варианты модификации протокола при помощи использования ЭЦП:
\begin{enumerate}
	\item Стороны  подписывают свои сообщения, посылаемые в ходе протокола, и проверяют подписи друг друга. Тогда им обоим понадобятся заверенные ЭЦП. 
	\item Подписывает только одна из сторон. Тогда двух сертификатов не понадобится.
\end{enumerate} 

	Проанализируем возможные сценарии при выполнении модифицированного протокола, чтобы убедиться, что он действительно защищён от атаки «человек посередине».\\
%%%%%%%%%%%%%%%%%%%%
%%%%%Протокол 5%%%%%
%%%%%%%%%%%%%%%%%%%%
\textbf{Протокол 5 (Шамира, подписывает только первая сторона)}\\
\hspace*{10mm}\textbf{Вход:} коммутативный шифр $\{X, Y, K, D, E\}$ для шифрования сеансового ключа, шифр $\{X', Y', K', D', E'\}$ для шифрования сообщений. Система ЭЦП $(S,V)$, где $S$ – алгоритм цифровой подписи, а $V$ – алгоритм проверки цифровой подписи.\\
Первая сторона: сеансовый ключ $s\in K',$ ключ $k_{1}\in K.$, алгоритм цифровой подписи $SA.$\\
Вторая сторона: ключ $k_{2}\in K$, алгоритм верификации цифровой подписи $VA.$\\
Злоумышленник: сеансовый ключ $s' \in K'$,  ключ $k1'\in K$, алгоритм цифровой подписи $S_{A'}$.\\
\hspace*{10mm}\textbf{Выход:} злоумышленник: ключ $s \in K'.$
\begin{enumerate}
	\item Первая сторона вычисляет $m_{1} = E_{k_{1}}(s)$ и посылает второй сообщение:
	$$m_{1,s} = (E_{k_{1}}(s), S_{A}(m_{1})).$$
	\begin{enumerate}
		\item Злоумышленник перехватывает $m_{1,s}$ и посылает первой стороне: 
		$$m_{2}'= E_{k_{2}'}(m1).$$
		\item Злоумышленник вычисляет $m_{1}' = E_{k_{1}'}(s')$ посылает второй стороне:
		$$m_{1,s'} = (m_{1}', S_{A'}(m_{1}')).$$
	\end{enumerate}
	\item Вторая сторона принимает $m_{1,s'}$, проверяет цифровую подпись алгоритмом $V_{A}$ и прекращает дальнейшее общение, так как подпись не пройдёт проверку.
	\item Первая сторона расшифровывает $m_{2}'$ и посылает второй:
	$$m_{3}'' = D_{k_{1}}(m_{2}').$$
	\begin{enumerate}
		\item Злоумышленник перехватывает $m_{3}''$ и получает $s = D_{k_{2}}(m_{3}'')$.
	\end{enumerate}
\end{enumerate}
%%%%%%%%%%%%%%%%%%%%
%%%%%Протокол 6%%%%%
%%%%%%%%%%%%%%%%%%%%
\textbf{Протокол 6 (Шамира, подписывает только вторая сторона)}\\
\hspace*{10mm}\textbf{Вход:} коммутативный шифр $\{X, Y, K, D, E\}$ для шифрования сеансового ключа, шифр $\{X', Y', K', D', E'\}$ для шифрования сообщений. Система ЭЦП $(S,V)$, где $S$ – алгоритм цифровой подписи, а $V$ – алгоритм проверки цифровой подписи.\\
Первая сторона: сеансовый ключ $s\in K',$ ключ $k_{1}\in K.$, алгоритм проверки цифровой подписи $S_{B}$.
Вторая сторона: ключ $k_{2}\in K$, алгоритм цифровой подписи $V_{B}$\\
Злоумышленник: сеансовый ключ $s' \in K'$,  ключ $k1'\in K$, алгоритм цифровой подписи $S_{B'}$.\\
\hspace*{10mm}\textbf{Выход:} вторая сторона: ключ $s'\in K'$.
\begin{enumerate}
\item Первая сторона посылает второй сообщение $m_{1} = E_{k_{1}}(s).$
\begin{enumerate}
	\item Злоумышленник перехватывает $m_{1}$, вычисляет: $m_{2}' = E_{k_{2}'}(m_{1})$ и посылает первой стороне: 
	$$m_{2,s}'= (m_{2}',S_{B'}(m_{2}')).$$
	\item Злоумышленник посылает второй стороне:
	$$m_{1} = E_{k_{1}'}(s').$$
\end{enumerate}
\item Вторая сторона принимает $m_{1}'$, вычисляет: $m_{2}'' = E_{k_{2}}(m_{1}')$ и посылает первой:
	$$m_{2,s}'' = (m_{2}'',S_{B}(m_{2}'')).$$
	\begin{enumerate}
		\item Злоумышленник перехватывает $m_{2,s}''$ и посылает второй стороне:
		$$m_{3}' = D_{k_{1}}(m_{2}'').$$
	\end{enumerate}
\item Первая сторона принимает $m_{2,s}'$, проверяет цифровую подпись $V_{B}$ и прекращает дальнейшее общение, так как подпись не пройдёт проверку.
\item Вторая сторона расшифровывает $s' = D_{k_{2}}(m_{3}').$
\end{enumerate}
%%%%%%%%%%%%%%%%%%%%
%%%%%Протокол 7%%%%%
%%%%%%%%%%%%%%%%%%%%
\textbf{Протокол 7 (Шамира, подписывают обе стороны)}\\
\hspace*{10mm}\textbf{Вход:} коммутативный шифр $\{X, Y, K, D, E\}$ для шифрования сеансового ключа, шифр $\{X', Y', K', D', E'\}$ для шифрования сообщений. Система ЭЦП $(S,V)$, где $S$ – алгоритм цифровой подписи, а $V$ – алгоритм проверки цифровой подписи.\\
Первая сторона: сеансовый ключ $s\in K',$ ключ $k_{1}\in K.$, алгоритм цифровой подписи $S_{A}$, алгоритм верификации цифровой подписи $V_{B}$.
Вторая сторона: ключ $k_{2}\in K$, алгоритм цифровой подписи $S_{B}$, алгоритм верификации цифровой подписи $V_{A}.$\\
Злоумышленник: сеансовый ключ $s' \in K'$,  ключ $k1'\in K$, алгоритм цифровой подписи $S_{A'}$, алгоритм цифровой подписи $S_{B'}$. \\
\hspace*{10mm}\textbf{Выход:} отсутствует.
\begin{enumerate}
	\item Первая сторона вычисляет $m_{1} = E_{k_{1}}(s)$ и посылает второй сообщение:
	$$m_{1,s} = (E_{k_{1}}(s), S_{A}(m_{1})).$$
	\begin{enumerate}
		\item Злоумышленник перехватывает $m_{1,s}$, вычисляет: $m_{2}' = E_{k_{2}'}(m1)$ и посылает первой стороне: 
		$$m_{2,s}'= (m_{2}',S_{B'}(m_{2}')).$$
		\item Злоумышленник вычисляет: $m_{1}' = (E_{k_{1}'}(s')$ и посылает второй стороне:
		$$m_{1,s}' = (m_{1}',S_{A'}(m_{1}')).$$
	\end{enumerate}
	\item Вторая сторона принимает $m_{1,s}'$,проверяет цифровую подпись алгоритмом $V_{A}$ и прекращает дальнейшее общение, так как подпись не пройдёт проверку.
	\item Первая сторона принимает $m_{2,s}'$, проверяет цифровую подпись $V_{B}$ и прекращает дальнейшее общение, так как подпись не пройдёт проверку.
\end{enumerate}
Из анализа возможных атак видно, что свою личность удостоверит та сторона, которая имеет цифровую подпись. Исходя из практических соображений, возможно применение различных сценариев. 

Пусть первая сторона – какой-нибудь общедоступный интернет-сервис, например, сервер времени, который отвечает на запрос значением текущего времени; а вторая – клиент. Тогда авторизация клиенту не нужна. Действительно, при перехвате его сообщений и дальнейшем выполнении протокола злоумышленником, последний получит доступ к общедоступному сервису. С другой стороны, авторизация серверу обязательна: в противном случае злоумышленник сможет сообщить клиенту неправильное время. 

Теперь предположим, что сервер – это сайт, который может совершать действия от имени клиента. Тогда уже авторизация нужна клиенту, иначе злоумышленник сможет действовать от его имени. А серверу авторизация необязательна, хотя и желательна: злоумышленник просто узнает, что клиент хочет совершить какие-то действия. 

Наконец, если важна авторазация и клиента, и сервера, то нужно будет использовать цифровые подписи на обеих сторонах. Такая ситуация возникает, например, при использовании почтового сервиса: одинакого недопустима и отправка, и получение поддельного письма.
\subsection{Преимущества модифицированного протокола Шамира перед аналогичными алгоритмами}
\begin{enumerate}
	\item Протокол Шамира, в отличие от протокола Диффи-Хеллмана, не просто генерирует ключ, а позволяет выбирать его первой стороне. Если первая захочет использовать в качестве ключа определённое, осмысленное слово, то с протоколом Шамира это возможно, при условии что длина ключа подходит для шифра защиты сеансового ключа.
	\item RSA тоже позволяет использовать произвольное слово. Но открытый ключ RSA меняется нечасто. Сеансовый ключ тоже может оставаться постоянным. Тогда шифрограммы, посылаемые первой стороной, будут всегда выглядеть одинаково, и криптоаналитик, сравнивая одинаковые сообщения, зашифрованные RSA, сможет получить о них некоторую информацию, например, количество ключей. При использовании протокола Шамира такого не произойдёт из-за того, что ключ для защиты сеансового ключа можно каждый раз выбирать произвольным образом.
	\item Описанная выше проблема с RSA разрешается добавлением к сообщению случайной строки – соли. Но это увеличит длину сообщений, что не годится в условиях жёстких ограничений на длину передаваемых данных.
	\item Если вдруг закрытый ключ RSA собеседника окажется скомпрометирован через какое-то время, то злоумышленник сможет получить ключи от старых сессий, расшифровав предыдущие сообщения, передаваемые в ходе протокола. С протоколом Шамира такой проблемы не возникнет, потому что сеансовый ключ каждый раз зашифровывается новыми случайными ключами.
\end{enumerate}
\subsection{Применение отображения <<Кошка Арнольда>> для построения коммутативного шифра}
Как уже было сказано, протоколу Шамира необходим коммутативный шифр. Предложим ещё один шифр, основанный на отображении <<Кошка Арнольда>>.

Пусть шифруемое сообщение $D$ представляет собой квадратное изображение, заданное набором координат точек $\{x,y\}$ и значениями их цветов. Разобьём его на квадраты величиной $50\times{}50$ точек: изображения такого размера отличаются большим периодом (\autocite{cat}, пункт 4), что даёт нам большее пространство ключей. В случае, если изображение нельзя разбить на целое количество таких квадратов, дополним его до необходимых параметров длины и ширины каким-нибудь шумом. 
Пусть в результате разбиения мы получили $k$ квадратов $D_{i}$ размера $50\times{}50$. Напомним, что при действии на каждый из них $\Gamma_{A}$ период отображения равен $150$. Выберем теперь $k$ случайных чисел от $0$ до $150$: $K = \{{m_{1},m_{2},\dots,m_{k}}\}$ и составим новое изображение $S$ из квадратов: 
$$S = \bigcup{}_{i = 1}^{k}S_{i} = \bigcup{}_{i = 1}^{k}\Gamma_{A}^{m_{i}}{D_{i}}.$$
Полученное изображение будет шифрограммой, а выбранная последовательность чисел --- ключом для расшифрования. Для зашифрования составим последовательность чисел: 
$$E = \{e_{1},e_{2},\dots{},e_{k}\} = 150 - K = \{150-m_{1},150-m_{2},\dots,150-m_{k}\}.$$
--- ключ для расшифрования. Составим новое изображенние $R$ из квадратов $R_{i}$:
$$R_{i} = \Gamma_{A}^{e_{i}}{S_{i}} =
\Gamma_{A}^{e_{i}}\Gamma_{A}^{m_{i}}{D_{i}} =
\Gamma_{A}^{e_{i}+m_{i}}{D_{i}} =
\Gamma_{A}^{150-m_{i}+m_{i}}{D_{i}} =
\Gamma_{A}^{150}{D_{i}} = 
D_{i.}$$
Последнее равенство справедливо ввиду того, что $150$ --- период $\Gamma_{A}$ при $N = 50$.

Необходимо показать, что шифр является коммутативнным. Пусть шифруется сообщение $D$. А пары ключей $K^{1} = \{{m_{1}^{1},m_{2}^{1},\dots,m_{k}^{1}}\}, E^{1} = \{{150 - m_{1}^{1},150 -m_{2}^{1},\dots,150 -m_{k}^{1}}\}$ и $K^{2} = \{{m_{1}^{2},m_{2}^{2},\dots,m_{k}^{2}}\}, E^{2} = \{150 -{m_{1}^{2},150 -m_{2}^{2},\dots,150 -m_{k}^{2}}\}$ выбраны как описано выше. Обозначения алгоритмов зашифрования и расшифрования --- стандартные. Покажем, что выполнено свойство коммутативности шифра, то есть:
$$
E_{K^{2}}((E_{K^{1}}(m)) = E_{K^{1}}((E_{K^{2}}(m))
$$
В самом деле:
\begin{gather*}
E_{K^{2}}((E_{K^{1}}(D)) = E_{K^{2}}(\bigcup{}_{i = 1}^{k}\Gamma_{A}^{m_{i}^{1}}{D_{i}}) = \\
=\bigcup{}_{i = 1}^{k}\Gamma_{A}^{m_{i}^{2}}(\Gamma_{A}^{m_{i}^{1}}{D_{i}}) 
\end{gather*}
Предыдущий переход справедлив, поскольку каждый раз изображение разбивается одинаковым образом. Далее:
\begin{gather*}
\bigcup{}_{i = 1}^{k}\Gamma_{A}^{m_{i}^{2}}(\Gamma_{A}^{m_{i}^{1}}{D_{i}})  = \bigcup{}_{i = 1}^{k}\Gamma_{A}^{m_{i}^{2}+m_{i}^{1}}{D_{i}} = \bigcup{}_{i = 1}^{k}\Gamma_{A}^{m_{i}^{1}+m_{i}^{2}}{D_{i}}=\\
= \bigcup{}_{i = 1}^{k}\Gamma_{A}^{m_{i}^{1}}(\Gamma_{A}^{m_{i}^{2}}{D_{i}}) =
E_{K^{1}}((E_{K^{2}}(D))
\end{gather*}
Таким образом, получили алгоритмы шифрования и расшифрования. 

Так как каждый квадрат $D_{i}$ можно зашифровать $151$ способами, то мощность пространства возможных сообщений равна $151^{k}$. Однако если необходимо зашифровать осмысленное изображение, то способов зашифровать каждый квадрат становится меньше, ввиду появления так называемых призраков и миниатюр --- уменьшенных копий изображения (\autocite{cat}, пункт 4.5).

% Печать списка литературы (библиографии)
\printbibliography[%{}
    heading=bibintoc%
    %,title=Библиография % если хочется это слово
]
% Файл со списком литературы: biblio.bib
% Подробно по оформлению библиографии:
% см. документацию к пакету biblatex-gost
% http://ctan.mirrorcatalogs.com/macros/latex/exptl/biblatex-contrib/biblatex-gost/doc/biblatex-gost.pdf
% и огромное количество примеров там же:
% http://mirror.macomnet.net/pub/CTAN/macros/latex/contrib/biblatex-contrib/biblatex-gost/doc/biblatex-gost-examples.pdf

\appendix
\ifthenelse{\value{worktype} > 1}{%
  \addtocontents{toc}{%
      \protect\renewcommand{\protect\cftchappresnum}{\appendixname\space}%
      \protect\addtolength{\protect\cftchapnumwidth}{\widthof{\appendixname\space{}} - \widthof{Глава }}%
  }%
}{
  \addtocontents{toc}{%
      \protect\renewcommand{\protect\cftsecpresnum}{\appendixname\space}%
      \protect\addtolength{\protect\cftsecnumwidth}{\widthof{\appendixname\space{}}}%
  }%
}

\end{document}
