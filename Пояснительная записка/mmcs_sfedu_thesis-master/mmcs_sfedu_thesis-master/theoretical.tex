
Также как протоколы RSA и Диффи-Хеллман, протокол Шамира уязвим к атаке человек посередине. 
\subsection{Способ защиты описанного протокола}
Источник уязвимости упомянутых протоколов кроется в том, что они не предоставляют авторизации, то есть у сторон нет никакого способа проверить, что они действительно общаются друг с другом, во время выполнения протокола. Для разрешения проблемы авторизации может быть использована заверенная ЭЦП.

Возможны различные варианты модификации протокола при помощи использования ЭЦП:
\begin{enumerate}
	\item Стороны  подписывают свои сообщения, посылаемые в ходе протокола, и проверяют подписи друг друга. Тогда им обоим понадобятся заверенные ЭЦП. 
	\item Подписывает только одна из сторон. Тогда двух сертификатов не понадобится.
\end{enumerate} 

	Проанализируем возможные сценарии при выполнении модифицированного протокола, чтобы убедиться, что он действительно защищён от атаки «человек посередине».\\
%%%%%%%%%%%%%%%%%%%%
%%%%%Протокол 5%%%%%
%%%%%%%%%%%%%%%%%%%%
\textbf{Протокол 5 (Шамира, подписывает только первая сторона)}\\
\hspace*{10mm}\textbf{Вход:} коммутативный шифр $\{X, Y, K, D, E\}$ для шифрования сеансового ключа, шифр $\{X', Y', K', D', E'\}$ для шифрования сообщений. Система ЭЦП $(S,V)$, где $S$ – алгоритм цифровой подписи, а $V$ – алгоритм проверки цифровой подписи.\\
Первая сторона: сеансовый ключ $s\in K',$ ключ $k_{1}\in K.$, алгоритм цифровой подписи $SA.$\\
Вторая сторона: ключ $k_{2}\in K$, алгоритм верификации цифровой подписи $VA.$\\
Злоумышленник: сеансовый ключ $s' \in K'$,  ключ $k1'\in K$, алгоритм цифровой подписи $S_{A'}$.\\
\hspace*{10mm}\textbf{Выход:} злоумышленник: ключ $s \in K'.$
\begin{enumerate}
	\item Первая сторона вычисляет $m_{1} = E_{k_{1}}(s)$ и посылает второй сообщение:
	$$m_{1,s} = (E_{k_{1}}(s), S_{A}(m_{1})).$$
	\begin{enumerate}
		\item Злоумышленник перехватывает $m_{1,s}$ и посылает первой стороне: 
		$$m_{2}'= E_{k_{2}'}(m1).$$
		\item Злоумышленник вычисляет $m_{1}' = E_{k_{1}'}(s')$ посылает второй стороне:
		$$m_{1,s'} = (m_{1}', S_{A'}(m_{1}')).$$
	\end{enumerate}
	\item Вторая сторона принимает $m_{1,s'}$, проверяет цифровую подпись алгоритмом $V_{A}$ и прекращает дальнейшее общение, так как подпись не пройдёт проверку.
	\item Первая сторона расшифровывает $m_{2}'$ и посылает второй:
	$$m_{3}'' = D_{k_{1}}(m_{2}').$$
	\begin{enumerate}
		\item Злоумышленник перехватывает $m_{3}''$ и получает $s = D_{k_{2}}(m_{3}'')$.
	\end{enumerate}
\end{enumerate}
%%%%%%%%%%%%%%%%%%%%
%%%%%Протокол 6%%%%%
%%%%%%%%%%%%%%%%%%%%
\textbf{Протокол 6 (Шамира, подписывает только вторая сторона)}\\
\hspace*{10mm}\textbf{Вход:} коммутативный шифр $\{X, Y, K, D, E\}$ для шифрования сеансового ключа, шифр $\{X', Y', K', D', E'\}$ для шифрования сообщений. Система ЭЦП $(S,V)$, где $S$ – алгоритм цифровой подписи, а $V$ – алгоритм проверки цифровой подписи.\\
Первая сторона: сеансовый ключ $s\in K',$ ключ $k_{1}\in K.$, алгоритм проверки цифровой подписи $S_{B}$.
Вторая сторона: ключ $k_{2}\in K$, алгоритм цифровой подписи $V_{B}$\\
Злоумышленник: сеансовый ключ $s' \in K'$,  ключ $k1'\in K$, алгоритм цифровой подписи $S_{B'}$.\\
\hspace*{10mm}\textbf{Выход:} вторая сторона: ключ $s'\in K'$.
\begin{enumerate}
\item Первая сторона посылает второй сообщение $m_{1} = E_{k_{1}}(s).$
\begin{enumerate}
	\item Злоумышленник перехватывает $m_{1}$, вычисляет: $m_{2}' = E_{k_{2}'}(m_{1})$ и посылает первой стороне: 
	$$m_{2,s}'= (m_{2}',S_{B'}(m_{2}')).$$
	\item Злоумышленник посылает второй стороне:
	$$m_{1} = E_{k_{1}'}(s').$$
\end{enumerate}
\item Вторая сторона принимает $m_{1}'$, вычисляет: $m_{2}'' = E_{k_{2}}(m_{1}')$ и посылает первой:
	$$m_{2,s}'' = (m_{2}'',S_{B}(m_{2}'')).$$
	\begin{enumerate}
		\item Злоумышленник перехватывает $m_{2,s}''$ и посылает второй стороне:
		$$m_{3}' = D_{k_{1}}(m_{2}'').$$
	\end{enumerate}
\item Первая сторона принимает $m_{2,s}'$, проверяет цифровую подпись $V_{B}$ и прекращает дальнейшее общение, так как подпись не пройдёт проверку.
\item Вторая сторона расшифровывает $s' = D_{k_{2}}(m_{3}').$
\end{enumerate}
%%%%%%%%%%%%%%%%%%%%
%%%%%Протокол 7%%%%%
%%%%%%%%%%%%%%%%%%%%
\textbf{Протокол 7 (Шамира, подписывают обе стороны)}\\
\hspace*{10mm}\textbf{Вход:} коммутативный шифр $\{X, Y, K, D, E\}$ для шифрования сеансового ключа, шифр $\{X', Y', K', D', E'\}$ для шифрования сообщений. Система ЭЦП $(S,V)$, где $S$ – алгоритм цифровой подписи, а $V$ – алгоритм проверки цифровой подписи.\\
Первая сторона: сеансовый ключ $s\in K',$ ключ $k_{1}\in K.$, алгоритм цифровой подписи $S_{A}$, алгоритм верификации цифровой подписи $V_{B}$.
Вторая сторона: ключ $k_{2}\in K$, алгоритм цифровой подписи $S_{B}$, алгоритм верификации цифровой подписи $V_{A}.$\\
Злоумышленник: сеансовый ключ $s' \in K'$,  ключ $k1'\in K$, алгоритм цифровой подписи $S_{A'}$, алгоритм цифровой подписи $S_{B'}$. \\
\hspace*{10mm}\textbf{Выход:} отсутствует.
\begin{enumerate}
	\item Первая сторона вычисляет $m_{1} = E_{k_{1}}(s)$ и посылает второй сообщение:
	$$m_{1,s} = (E_{k_{1}}(s), S_{A}(m_{1})).$$
	\begin{enumerate}
		\item Злоумышленник перехватывает $m_{1,s}$, вычисляет: $m_{2}' = E_{k_{2}'}(m1)$ и посылает первой стороне: 
		$$m_{2,s}'= (m_{2}',S_{B'}(m_{2}')).$$
		\item Злоумышленник вычисляет: $m_{1}' = (E_{k_{1}'}(s')$ и посылает второй стороне:
		$$m_{1,s}' = (m_{1}',S_{A'}(m_{1}')).$$
	\end{enumerate}
	\item Вторая сторона принимает $m_{1,s}'$,проверяет цифровую подпись алгоритмом $V_{A}$ и прекращает дальнейшее общение, так как подпись не пройдёт проверку.
	\item Первая сторона принимает $m_{2,s}'$, проверяет цифровую подпись $V_{B}$ и прекращает дальнейшее общение, так как подпись не пройдёт проверку.
\end{enumerate}
Из анализа возможных атак видно, что свою личность удостоверит та сторона, которая имеет цифровую подпись. Исходя из практических соображений, возможно применение различных сценариев. 

Пусть первая сторона – какой-нибудь общедоступный интернет-сервис, например, сервер времени, который отвечает на запрос значением текущего времени; а вторая – клиент. Тогда авторизация клиенту не нужна. Действительно, при перехвате его сообщений и дальнейшем выполнении протокола злоумышленником, последний получит доступ к общедоступному сервису. С другой стороны, авторизация серверу обязательна: в противном случае злоумышленник сможет сообщить клиенту неправильное время. 

Теперь предположим, что сервер – это сайт, который может совершать действия от имени клиента. Тогда уже авторизация нужна клиенту, иначе злоумышленник сможет действовать от его имени. А серверу авторизация необязательна, хотя и желательна: злоумышленник просто узнает, что клиент хочет совершить какие-то действия. 

Наконец, если важна авторазация и клиента, и сервера, то нужно будет использовать цифровые подписи на обеих сторонах. Такая ситуация возникает, например, при использовании почтового сервиса: одинакого недопустима и отправка, и получение поддельного письма.
\subsection{Преимущества модифицированного протокола Шамира перед аналогичными алгоритмами}
\begin{enumerate}
	\item Протокол Шамира, в отличие от протокола Диффи-Хеллмана, не просто генерирует ключ, а позволяет выбирать его первой стороне. Если первая захочет использовать в качестве ключа определённое, осмысленное слово, то с протоколом Шамира это возможно, при условии что длина ключа подходит для шифра защиты сеансового ключа.
	\item RSA тоже позволяет использовать произвольное слово. Но открытый ключ RSA меняется нечасто. Сеансовый ключ тоже может оставаться постоянным. Тогда шифрограммы, посылаемые первой стороной, будут всегда выглядеть одинаково, и криптоаналитик, сравнивая одинаковые сообщения, зашифрованные RSA, сможет получить о них некоторую информацию, например, количество ключей. При использовании протокола Шамира такого не произойдёт из-за того, что ключ для защиты сеансового ключа можно каждый раз выбирать произвольным образом.
	\item Описанная выше проблема с RSA разрешается добавлением к сообщению случайной строки – соли. Но это увеличит длину сообщений, что не годится в условиях жёстких ограничений на длину передаваемых данных.
	\item Если вдруг закрытый ключ RSA собеседника окажется скомпрометирован через какое-то время, то злоумышленник сможет получить ключи от старых сессий, расшифровав предыдущие сообщения, передаваемые в ходе протокола. С протоколом Шамира такой проблемы не возникнет, потому что сеансовый ключ каждый раз зашифровывается новыми случайными ключами.
\end{enumerate}
\subsection{Применение отображения <<Кошка Арнольда>> для построения коммутативного шифра}
Как уже было сказано, протоколу Шамира необходим коммутативный шифр. Предложим ещё один шифр, основанный на отображении <<Кошка Арнольда>>.

Пусть шифруемое сообщение $D$ представляет собой квадратное изображение, заданное набором координат точек $\{x,y\}$ и значениями их цветов. Разобьём его на квадраты величиной $50\times{}50$ точек: изображения такого размера отличаются большим периодом (\autocite{cat}, пункт 4), что даёт нам большее пространство ключей. В случае, если изображение нельзя разбить на целое количество таких квадратов, дополним его до необходимых параметров длины и ширины каким-нибудь шумом. 
Пусть в результате разбиения мы получили $k$ квадратов $D_{i}$ размера $50\times{}50$. Напомним, что при действии на каждый из них $\Gamma_{A}$ период отображения равен $150$. Выберем теперь $k$ случайных чисел от $0$ до $150$: $K = \{{m_{1},m_{2},\dots,m_{k}}\}$ и составим новое изображение $S$ из квадратов: 
$$S = \bigcup{}_{i = 1}^{k}S_{i} = \bigcup{}_{i = 1}^{k}\Gamma_{A}^{m_{i}}{D_{i}}.$$
Полученное изображение будет шифрограммой, а выбранная последовательность чисел --- ключом для расшифрования. Для зашифрования составим последовательность чисел: 
$$E = \{e_{1},e_{2},\dots{},e_{k}\} = 150 - K = \{150-m_{1},150-m_{2},\dots,150-m_{k}\}.$$
--- ключ для расшифрования. Составим новое изображенние $R$ из квадратов $R_{i}$:
$$R_{i} = \Gamma_{A}^{e_{i}}{S_{i}} =
\Gamma_{A}^{e_{i}}\Gamma_{A}^{m_{i}}{D_{i}} =
\Gamma_{A}^{e_{i}+m_{i}}{D_{i}} =
\Gamma_{A}^{150-m_{i}+m_{i}}{D_{i}} =
\Gamma_{A}^{150}{D_{i}} = 
D_{i.}$$
Последнее равенство справедливо ввиду того, что $150$ --- период $\Gamma_{A}$ при $N = 50$.

Необходимо показать, что шифр является коммутативнным. Пусть шифруется сообщение $D$. А пары ключей $K^{1} = \{{m_{1}^{1},m_{2}^{1},\dots,m_{k}^{1}}\}, E^{1} = \{{150 - m_{1}^{1},150 -m_{2}^{1},\dots,150 -m_{k}^{1}}\}$ и $K^{2} = \{{m_{1}^{2},m_{2}^{2},\dots,m_{k}^{2}}\}, E^{2} = \{150 -{m_{1}^{2},150 -m_{2}^{2},\dots,150 -m_{k}^{2}}\}$ выбраны как описано выше. Обозначения алгоритмов зашифрования и расшифрования --- стандартные. Покажем, что выполнено свойство коммутативности шифра, то есть:
$$
E_{K^{2}}((E_{K^{1}}(m)) = E_{K^{1}}((E_{K^{2}}(m))
$$
В самом деле:
\begin{gather*}
E_{K^{2}}((E_{K^{1}}(D)) = E_{K^{2}}(\bigcup{}_{i = 1}^{k}\Gamma_{A}^{m_{i}^{1}}{D_{i}}) = \\
=\bigcup{}_{i = 1}^{k}\Gamma_{A}^{m_{i}^{2}}(\Gamma_{A}^{m_{i}^{1}}{D_{i}}) 
\end{gather*}
Предыдущий переход справедлив, поскольку каждый раз изображение разбивается одинаковым образом. Далее:
\begin{gather*}
\bigcup{}_{i = 1}^{k}\Gamma_{A}^{m_{i}^{2}}(\Gamma_{A}^{m_{i}^{1}}{D_{i}})  = \bigcup{}_{i = 1}^{k}\Gamma_{A}^{m_{i}^{2}+m_{i}^{1}}{D_{i}} = \bigcup{}_{i = 1}^{k}\Gamma_{A}^{m_{i}^{1}+m_{i}^{2}}{D_{i}}=\\
= \bigcup{}_{i = 1}^{k}\Gamma_{A}^{m_{i}^{1}}(\Gamma_{A}^{m_{i}^{2}}{D_{i}}) =
E_{K^{1}}((E_{K^{2}}(D))
\end{gather*}
Таким образом, получили алгоритмы шифрования и расшифрования. 

Так как каждый квадрат $D_{i}$ можно зашифровать $151$ способами, то мощность пространства возможных сообщений равна $151^{k}$. Однако если необходимо зашифровать осмысленное изображение, то способов зашифровать каждый квадрат становится меньше, ввиду появления так называемых призраков и миниатюр --- уменьшенных копий изображения (\autocite{cat}, пункт 4.5).
